\documentclass[9pt,twocolumn,twoside]{styles/osajnl} 
\usepackage{geometry}
\geometry{paper=letterpaper,margin=2cm}


\usepackage{xcolor}
\usepackage{fancyvrb}
\journal{cm}  
\usepackage{listings}
\usepackage{minted}
\usepackage[most]{tcolorbox}
\tcbuselibrary{minted}

\definecolor{codebggray}{rgb}{0.95,0.95,0.95}



\newcounter{filePrg}


\tcbuselibrary{listings}
\tcbuselibrary{minted}

\newtcbinputlisting[use counter=filePrg, number within=section, list inside=mypyg]{\codeFromFile}[4]{%
  listing engine=minted,
  minted language=#1,
  listing file={#2},
  minted options={autogobble,linenos,breaklines},
  listing only,
  size=title,
    drop fuzzy shadow,
  breakable,
  enhanced jigsaw,
    %arc=1.5mm,
  %colframe=brown,
  %coltitle=White,
  %boxrule=0.5mm,
  %colback=white,
  %coltext=Black,
  title={Listing \thetcbcounter : #3 \hfill%
    \smash{\raisebox{-6pt}{\includegraphics[width=.5cm,height=.5cm]{images/code2.png}}}},
    list entry=Listing~\thetcbcounter : #3,
    %OR next two lines
    %title={Listing \thetcbcounter : #3},
    %overlay={\node[anchor=north east,outer sep=-9pt] at ([xshift=-25pt]frame.north east) {\includegraphics[width=1cm,height=1cm]{images/code2.png}}; },
  label=lst:#4
}

%http://tex.stackexchange.com/questions/217489/numbering-tcolorbox-toc
\makeatletter % no indent for entries
\renewcommand{\l@tcolorbox}{\@dottedtocline{1}{0pt}{2.3em}}
\makeatother

\usepackage{hyperref}
\usepackage[ngerman,nameinlink]{cleveref}

\crefname{filePrg}{Listing}{Listings}


\newtcblisting{json}[1][]{%
  colback=codebggray,
  listing only,
  minted options={
    fontsize=\small
  },  
  minted language=json,
  #1
}

\newcommand{\tightlist}{}

\title{Cloudmesh REST Interface for Virtual Clusters} 

\author[1,*]{Gregor von Laszewski} 
\author[1]{Fugang Wang}
\author[1]{Badi Abdhul-Wahid}

\affil[1]{School of Informatics and Computing, Bloomington, IN 47408, U.S.A.} 
\affil[*]{Corresponding authors: laszewski@gmal.com} 

\dates{Draft v0.0.1, \today} 

\ociscodes{CLoudmesh, REST, NIST} 

\doi{\url{https://github.com/cloudmesh/rest/tree/master/docs}} 


\begin{abstract}

This document summarizes a number of objects that are instrumental for
the interaction with Clouds, Containers, and HPC systems to manage
virtual clusters. 
TBD

\end{abstract}

\begin{document}


\flushbottom % Makes all text pages the same height

\maketitle % Print the title and abstract box

\tableofcontents % Print the contents section
\maketitle

%%%%%%%%%%%%%%%%%%%%%%%%%%%%%%%%%%%%%%%%%%%%%%%%%%%%%%%%%%%%%%%%%%%%%%
\section{Introduction}
%%%%%%%%%%%%%%%%%%%%%%%%%%%%%%%%%%%%%%%%%%%%%%%%%%%%%%%%%%%%%%%%%%%%%%

\subsection{Design by Example}

To accelereat discussion among the team we use an approach to define
objects and its interfaces by  example. These examples are than taken
in a later version of the document and a schema is generated from
it. The schema will be added in its complete form to the
appendix. While focusing first on examples it allows us to speed up
our design and simplifies discussions of the objects and interfaces
eliminating getting lost in complex syntactical specifications. The
process and specifications used in this document will also allow us to
automatically create a implementation of the objects that can be
integrated into a reference architecture as provided by for example
the cloudmesh client and rest project \cite{??}.  

An example object will demonstrate our approach. The following object
defines a JSON object representing a user. 

\begin{json}
"user":{  
     "firstname": "Gregor",  
     "lastname": "von Laszewski",  
     "email": "laszewski@gmail.com",  
     "username": "gregor" 
}
\end{json}


\begin{Verbatim}
"user":{  
     "firstname": "Gregor",  
     "lastname": "von Laszewski",  
     "email": "laszewski@gmail.com",  
     "username": "gregor" 
}
\end{Verbatim}
 
Such an object can be transformed to a more precise specification as a
schema while introspecting the types resulting in  

\begin{Verbatim}
user = {  
   ’firstname’: { ’type’: ’string’},  
   ’lastname’: { ’type’: ’string’ },   
   ’email’: { ’type’: ’email’ },  
   ’username’: { ’type’: ‘string’ } 
}  
\end{Verbatim}

The appendix will list the examples as well as the schema definitions
(see appendix ???).  

\subsection{Tools to Create the Specifications}

\begin{Verbatim}
cloudmesh/evegenie
cloudmesh.cmd5
cloudmesh.rest
\end{Verbatim}


\subsection{Interface Compliancy}

Due to the extensibility of our interfaces it is important to
introduce a terminology that allows us to define interface
compliancy. We define it as follows

\begin{description}

\item[Full Compliance:] These are reference implementations that
  provide full compliance to the objects defined in this document. A
  version number will be added to assure the snapshot in time of the
  objects is associated with the version. This reference
  implementation will implement all objects.

\item[Partially Compliance:] These are reference implementations that
  provide partial compliance to the objects defined in this
  document. A version number will be added to assure the snapshot in
  time of the objects is associated with the version. This reference
  implementation will implement a partial list of the objects. A
  document is accompanied that lists all objects defined, but also
  lists the objects that are not defined by the reference
  architecture.

\item[Full and extended Compliance:] These are interfaces that in
  addition to the full compliance also introduce additional interfaces
  and extend them.

\end{description}

%%%%%%%%%%%%%%%%%%%%%%%%%%%%%%%%%%%%%%%%%%%%%%%%%%%%%%%%%%%%%%%%%%%%%%
\section{User Profile}
%%%%%%%%%%%%%%%%%%%%%%%%%%%%%%%%%%%%%%%%%%%%%%%%%%%%%%%%%%%%%%%%%%%%%%

\subsection{profile}


\codeFromFile{json}{../cloudmesh/specification/examples/profile.json}{User profile}{s:profile}

\subsection{user}

\codeFromFile{json}{../cloudmesh/specification/examples/user.json}{user}{s:user}


%%%%%%%%%%%%%%%%%%%%%%%%%%%%%%%%%%%%%%%%%%%%%%%%%%%%%%%%%%%%%%%%%%%%%%
\section{Data}
%%%%%%%%%%%%%%%%%%%%%%%%%%%%%%%%%%%%%%%%%%%%%%%%%%%%%%%%%%%%%%%%%%%%%%

Data for Big Data applications are delivered through data
providers. They can be either local providers contributed by a user or
distributed data providers that refer to data on the internet. At this
time we focus on an elementary set of abstractions related to data
providers that offer us to utilize variables, files, virtual data
directories, data streams, and data filters.

\begin{description}
\item[Files] are used to represent information collected within the context
of classical files in an operating system. 

\item[Streams] are services that offer the consumer a stream of data. Streams may allow the initiation of filters to reduce the amount of data requested by the consumer 
Stream Filters operate in streams or on files converting them to streams 

\item[Batch Filters] operate on streams and on files while working in
  the background and delivering as output Files 

\item[Virtual directories] and non-virtual directories are collection
  of files that organize them. For our initial purpose the distinction
  between virtual and non-virtual directories is non-essential and we
  will focus on abstracting all directories to be virtual. This could
  mean that the files are physically hosted on different
  disks. However, it is important to note that virtual data
  directories can hold more than files, they can also contain data
  streams and data filters.

\end{description}

\subsection{Var}

\codeFromFile{json}{../cloudmesh/specification/examples/var.json}{var}{s:var}

\subsection{Default}

\codeFromFile{json}{../cloudmesh/specification/examples/default.json}{default}{s:default}

\subsection{File}

A file is a computer resource allowing to store data that is being
processed. The interface to a file provides the mechanism to
appropriately locate a file in a distributed system. Identification
include the name, and andpoint, the checksum and the size. Additional
parameters such as the lasst access time could be stored also. As such
the Interface only describes the location of the file 

The \textbf{file} object has \textit{name}, \textit{endpoint} (location), \textit{size}
in GB, MB, Byte, \textit{checksum} for integrity check, and last
\textit{accessed} timestamp. 

\codeFromFile{json}{../cloudmesh/specification/examples/file.json}{file}{s:file}

\subsection{File Alias}

A file could have one alias or even multiple ones.

\codeFromFile{json}{../cloudmesh/specification/examples/file_alias.json}{file alias}{s:file-alias}

\subsection{Replica}

In many distributed systems, it is of importance that a file can be
replicated among different systems in order to provide faster
access. It is important to provide a mechanism that allows to trace
the pedigree of the file while pointing to its original source 

\codeFromFile{json}{../cloudmesh/specification/examples/replica.json}{replica}{s:replica}


\subsection{Virtual Directory}

A collection of files or replicas. A virtual directory can contain an
number of entities cincluding files, streams, and other virtual
directories as part of a collection. The element in the collection can
either be defined by uuid or by name. 

\codeFromFile{json}{../cloudmesh/specification/examples/virtual_directory.json}{virtual
  directory}{s:virtual-directory}

\subsection{Database}

A \textbf{database} could have a name, an \textit{endpoint} (e.g., host:port),
and protocol used (e.g., SQL, mongo, etc.).

\codeFromFile{json}{../cloudmesh/specification/examples/database.json}{database}{s:database}

\subsection{Stream} 

<Description> 
 
Stream source 
 
Stream subscriber 
 
Values: 
Rate 
Limit 
Redirect (virtual/alias) 
Pipe 
Filters 
 
Functions: 
list 
Subscribe 
Unsubscribe 
Get 
%%%%%%%%%%%%%%%%%%%%%%%%%%%%%%%%%%%%%%%%%%%%%%%%%%%%%%%%%%%%%%%%%%%%%%
\section{IaaS}
%%%%%%%%%%%%%%%%%%%%%%%%%%%%%%%%%%%%%%%%%%%%%%%%%%%%%%%%%%%%%%%%%%%%%%


% ----------------------------------------------------------------------
\subsection{Openstack}
% ----------------------------------------------------------------------

\subsubsection{Openstack Flavor}

\codeFromFile{json}{../cloudmesh/specification/examples/openstack_flavor.json}{openstack flavor}{s:openstack-flavor}

\subsubsection{Openstack Image}

\codeFromFile{json}{../cloudmesh/specification/examples/openstack_image.json}{openstack
  image}{s:openstack-image}

\subsubsection{Openstack Vm}

\codeFromFile{json}{../cloudmesh/specification/examples/openstack_vm.json}{openstack
  vm}{s:openstack-vm}

% ----------------------------------------------------------------------
\subsection{Azure}
% ----------------------------------------------------------------------

\subsubsection{Azure Size}

The size description of an azure vm

\codeFromFile{json}{../cloudmesh/specification/examples/azure-size.json}{azure-size}{s:azure-size}


\subsubsection{Azure Image}

\codeFromFile{json}{../cloudmesh/specification/examples/azure-image.json}{azure-image}{s:azure-image}

\subsubsection{Azure Vm}

An Azure virtual machine
\codeFromFile{json}{../cloudmesh/specification/examples/azure-vm.json}{azure-vm}{s:azure-vm}



%%%%%%%%%%%%%%%%%%%%%%%%%%%%%%%%%%%%%%%%%%%%%%%%%%%%%%%%%%%%%%%%%%%%%%
\section{HPC}
%%%%%%%%%%%%%%%%%%%%%%%%%%%%%%%%%%%%%%%%%%%%%%%%%%%%%%%%%%%%%%%%%%%%%%

\subsection{Batch Job}

\codeFromFile{json}{../cloudmesh/specification/examples/batchjob.json}{batchjob}{s:batchjob}

%%%%%%%%%%%%%%%%%%%%%%%%%%%%%%%%%%%%%%%%%%%%%%%%%%%%%%%%%%%%%%%%%%%%%%
\section{Virtual Cluster}
%%%%%%%%%%%%%%%%%%%%%%%%%%%%%%%%%%%%%%%%%%%%%%%%%%%%%%%%%%%%%%%%%%%%%%

\subsection{Cluster}

The cluster object has name, label, endpoint and provider. The
\textit{endpoint} defines....  The \textit{provider} defines the
nature of the cluster, e.g., its a virtual cluster on an openstack
cloud, or from AWS, or a bare-metal cluster.


\codeFromFile{json}{../cloudmesh/specification/examples/cluster.json}{cluster}{s:cluster}
\
\subsection{Compute Resource}

An important concept for big data analysis it the representation of a
compute resource on which we execute the analysis. We define a compute
resource by name and by endpoint. A compute resource is an abstract
concept and can be instantiated through virtual machines, containers,
or bare metal resources. This is defined by the “kind” of the compute
resource 

\textbf{compute\_resource} object has attribute \textit{endpoint} which
specifies ... The \textit{kind} could be \textit{baremetal} or \textit{VC}.

\codeFromFile{json}{../cloudmesh/specification/examples/compute_resource.json}{compute resource}{s:compute-resource}

\subsection{Computer}

This defines a \textbf{computer} object. A computer has name, label,
IP address. It also listed the relevant specs such as memory, disk
size, etc.

\codeFromFile{json}{../cloudmesh/specification/examples/computer.json}{computer}{s:computer}


\subsection{node}

A node is composed of multiple components:

\begin{enumerate}
\item Metadata such as the \verb|name| or \verb|owner|.
\item Physical properties such as \verb|cores| or \verb|memory|.
\item Configuration guidance such as \verb|create_external_ip|,
  \verb|security_groups|, or \verb|users|.
\end{enumerate}

The metadata is associated with the node on the provider end (if
supported) as well as in the database. Certain parts of the metadata
(such as \verb|owner|) can be used to implement access
control. Physical properties are relevant for the initial allocation
of the node. Other configuration parameters control and further
provisioning.

In the above, after allocation, the node is configured with a user
called \verb|hello| who is part of the \verb|wheel| group whose
account can be accessed with several SSH identities whose public keys
are provided (in \verb|authorized_keys|).

Additionally, three ssh keys are generated on the node for the
\verb|hello| user. The first uses the \verb|ed25519| cryptographic
method with a password read in from a GPG-encrypted file on the
Command and Control node. The second is a 4098-bit RSA key also
password-protected from the GPG-encrypted file. The third key is
copied to the remote node from an encrypted file on the Command and
Control node.

This definition also provides a security group to control access to
the node from the wide-area-network. In this case all ingress and
egress TCP and UDP traffic is allowed provided they are to ports 22
(SSH), 443 (SSL), and 80 and 8080 (web).


\codeFromFile{json}{../cloudmesh/specification/examples/node-new.json}{node-new}{s:node-new}



\subsection{Virtual Cluster}

A virtual cluster is an agglomeration of virtual compute nodes that
constitute the cluster. Nodes can be assembled to be baremetal,
virtual machines, and containers. A virtual cluster contains a number
of virtual compute nodes.  
 
\codeFromFile{json}{../cloudmesh/specification/examples/virtual_cluster.json}{virtual
  cluster}{s:virtual-cluster}

\subsection{Virtual Compute node}

\codeFromFile{json}{../cloudmesh/specification/examples/virtual_compute_node.json}{virtual
  compute node}{s:virtual-compute-node}

\subsection{Virtual Machine}

Virtual Machine 
Virtual machines are an emulation of a computer system. We are maintaining a very basic set of information. It is expected that through the endpoint the virtual machine can be introspected and more detailed information can be retrieved. 

\codeFromFile{json}{../cloudmesh/specification/examples/virtual_machine.json}{virtual machine}{s:virtual-machine}

\subsection{Mesos}

\codeFromFile{json}{../cloudmesh/specification/examples/mesos.json}{mesos}{s:mesos}

%%%%%%%%%%%%%%%%%%%%%%%%%%%%%%%%%%%%%%%%%%%%%%%%%%%%%%%%%%%%%%%%%%%%%%
\section{Containers}
%%%%%%%%%%%%%%%%%%%%%%%%%%%%%%%%%%%%%%%%%%%%%%%%%%%%%%%%%%%%%%%%%%%%%%

\subsection{Container}

This defines \textbf{container} object.

\codeFromFile{json}{../cloudmesh/specification/examples/container.json}{container}{s:container}

\subsection{Kubernetes}

\codeFromFile{json}{../cloudmesh/specification/examples/kubernetes.json}{kubernetes}{s:kubernetes}

%%%%%%%%%%%%%%%%%%%%%%%%%%%%%%%%%%%%%%%%%%%%%%%%%%%%%%%%%%%%%%%%%%%%%%
\section{Deployment}
%%%%%%%%%%%%%%%%%%%%%%%%%%%%%%%%%%%%%%%%%%%%%%%%%%%%%%%%%%%%%%%%%%%%%%

\subsection{Deployment}

A \textbf{deployment} consists of the resource \- \textit{cluster},
the location \- \textit{provider}, e.g., AWS, OpenStack, etc., and
software \textit{stack} to be deployed (e.g., hadoop, spark).

\codeFromFile{json}{../cloudmesh/specification/examples/deployment.json}{deployment}{s:deployment}

%%%%%%%%%%%%%%%%%%%%%%%%%%%%%%%%%%%%%%%%%%%%%%%%%%%%%%%%%%%%%%%%%%%%%%
\section{Mapreduce}
%%%%%%%%%%%%%%%%%%%%%%%%%%%%%%%%%%%%%%%%%%%%%%%%%%%%%%%%%%%%%%%%%%%%%%

\subsection{Hadoop}

A \textbf{hadoop} definition defines which \textit{deployer} to be used,
the \textit{parameters} of the deployment, and the system packages as
\textit{requires}. For each requirement, it could have attributes such
as the library origin, version, etc.

\codeFromFile{json}{../cloudmesh/specification/examples/hadoop.json}{hadoop}{s:hadoop}


\subsection{Mapreduce}

This defines a \textbf{mapreduce} deployment with its layered
components.

\codeFromFile{json}{../cloudmesh/specification/examples/mapreduce.json}{mapreduce}{s:mapreduce}

%%%%%%%%%%%%%%%%%%%%%%%%%%%%%%%%%%%%%%%%%%%%%%%%%%%%%%%%%%%%%%%%%%%%%%
\section{Security}
%%%%%%%%%%%%%%%%%%%%%%%%%%%%%%%%%%%%%%%%%%%%%%%%%%%%%%%%%%%%%%%%%%%%%%

\subsection{Key}

\codeFromFile{json}{../cloudmesh/specification/examples/key.json}{key}{s:key}

%%%%%%%%%%%%%%%%%%%%%%%%%%%%%%%%%%%%%%%%%%%%%%%%%%%%%%%%%%%%%%%%%%%%%%
\section{Microservice}
%%%%%%%%%%%%%%%%%%%%%%%%%%%%%%%%%%%%%%%%%%%%%%%%%%%%%%%%%%%%%%%%%%%%%%

\subsection{Microservice}

 
introduce registry we can register many things to it 
latency 
provide example on how to use each of them, not just the object definition example 
 
necessity of local direct attached storage. 
Mimd model to storage  
Kubernetis, mesos can not spin up ?  
Takes time to spin them up and coordinate them. While setting up environment takes more thsn using the microservice, so we must make sure that the micorservices are used sufficiently to offset spinup cost. 
 
limitation of resource capacity such as networking. 
 
Benchmarking to find out thing about service level agreement to access
the 


A system could be composed of from various microservices, and this defines
each of them.

\codeFromFile{json}{../cloudmesh/specification/examples/microservice.json}{microservice}{s:microservice}


%%%%%%%%%%%%%%%%%%%%%%%%%%%%%%%%%%%%%%%%%%%%%%%%%%%%%%%%%%%%%%%%%%%%%%
\subsection{Reservation}
%%%%%%%%%%%%%%%%%%%%%%%%%%%%%%%%%%%%%%%%%%%%%%%%%%%%%%%%%%%%%%%%%%%%%%

\codeFromFile{json}{../cloudmesh/specification/examples/reservation.json}{reservation}{s:reservation}

%%%%%%%%%%%%%%%%%%%%%%%%%%%%%%%%%%%%%%%%%%%%%%%%%%%%%%%%%%%%%%%%%%%%%%
\section{Network}
%%%%%%%%%%%%%%%%%%%%%%%%%%%%%%%%%%%%%%%%%%%%%%%%%%%%%%%%%%%%%%%%%%%%%%

We are looking for volunteers to contribute here.

\newpage

\appendix

\section{Schema Command}

%\codeFromFile{txt}{schema-man.tex}{man page}{s:manpage}


\section{Contributing}

We invite you to contribute to this paper and its discussion to
improve it. Improvements can be done with pull requests. We suggest
you do {\em small} individual changes to a single section and object
rather than large changes as this allows us to integrate the changes
individually and comment on your contribution via github.

Once contributed we will appropriately acknoledge you either as
contributor or author. Please discuss with us how we best acknowledge
you.

\section{Using the Cloudmesh REST Service} 

Components are written as YAML markup in files in the
\verb+resources/samples+ directory.

For example:

\codeFromFile{json}{../cloudmesh/specification/examples/profile.json}{profile}{s:profile}

\subsection{Element Definition}

Each resource should have a \verb+description+ entry to act as
documentation. The documentation should be formated as
reStructuredText. For example:

\subsection{Yaml}

\begin{Verbatim}
entry = yaml.read('''
profile:
  description: |
    A user profile that specifies general information 
    about the user
  email: laszewski@gmail.com, required
  firtsname: Gregor, required
  lastname: von Laszewski, required
  height: 180
'''}
\end{Verbatim}

\subsection{Cerberus}

\begin{Verbatim}
schema = {
'profile': {
  'description': {'type': 'string'}
  'email':       {'type': 'string', 'required': True}
  'firtsname':   {'type': 'string', 'required': True}
  'lastname':    {'type': 'string', 'required': True}
  'height':      {'type': 'float'}
}
\end{Verbatim}

\section{Mongoengine}

\begin{Verbatim}
class profile(Document):
    description = StringField()
    email = EmailField(required=True)
    firstname = StringField(required=True)
    lastname = StringField(required=True)
    height = FloatField(max_length=50)
\end{Verbatim}

\section{Cloudmesh Notation}

\begin{Verbatim}
profile:
    description: string
    email: email, required
    firstname: string, required
    lastname: string, required
    height: flat, max=10
\end{Verbatim}

\begin{Verbatim}
proposed command

cms schema FILENAME --format=mongo -o OUTPUT
cms schema FILENAME --format=cerberus -o OUTPUT
cms schema FILENAME --format=yaml -o OUTPUT

  reads FILENAME in cloudmesh notation and returns format


cms schema FILENAME --input=evegenie -o OUTPUT
   reads eavegene example and create settings for eve
\end{Verbatim}


\subsection{Defining Elements for the REST Service}

To manage a large number of elements defined in our REST service
easily, we manage them trhough definitions in yaml files. To generate
the appropriate settings file for the rest service, we can use teh
following command:

\begin{verbatim}
cms admin elements <directory> <out.json>
\end{verbatim}

where

\begin{itemize}
\item \verb+<directory>+: directory where the YAML definitions reside
\item \verb+<out.json>+: path to the combined definition
\end{itemize}

For example, to generate a file called all.json that integrates all
yml objects defined in the directory \verb+resources/samples+ you can
use the following command:

\begin{verbatim}
cms elements resources/samples all.json
\end{verbatim}

\subsection{DOIT}


cms schema spec2tex resources/specification resources/tex

\subsection{Generating service}

With evegenie installed, the generated JSON file from the above step
is processed to create the stub REST service definitions.


\section{ABC}
{\bf README.rst}
\section{Cloudmesh Rest}\label{s:cloudmesh-rest}

Cloudmesh Resst is a refernce implementattion for the NBDRA. It
allows to define automatically a REST service based on the objects
specified by the NBDRA document. In collaboration with other cloudmesh
components it allows easy interaction with hybrid clouds and the
creation of user managed big data services. 

\subsection{Prerequistis}\label{prerequistis}

The preriquisits for Cloudmesh REST are Python 2.7.13 or 3.6.1
it can easily be installed on a variety of systems (at this time we
have only tried ubuntu greater 16.04 and OSX Sierra. However, it would
naturally be possible to also port it to Windows. The instalation
instruction in this document are not complete and we recommend to
refer to the cloudmesh manuals which are under development. The goal
will be to make the instalation (after your system is set up for
developing python) as simple as 

\begin{verbatim}
    pip install cloudmesh.rest
\end{verbatim}


\subsection{REST Service}\label{cm-rest}

With the cloudmesh REST framework it is easy to create REST services
while defining the resources via example json objects. This is achieved
while leveraging the python eve \cite{www-eve} and a modified version of python
evengine \cite{www-cloudmesh-eveengine}. 

A valid json resource specification looks like this:

\begin{verbatim}
{
  "profile": {
    "description": "The Profile of a user",
    "email": "laszewski@gmail.com",
    "firstname": "Gregor",
    "lastname": "von Laszewski",
    "username": "gregor"
  }
}
\end{verbatim}

here we define an object called profile, that contains a number of
attributes and values. The type of the values are automatically
determined. All json specifications are contained in a directory and
can easily be converted into a valid schema for the eve rest service
by executing the commands

\begin{verbatim}
cms schema cat . all.json
cms schema convert all.json
\end{verbatim}

This will create a the configuration \verb|all.settings.py| that can
be used to start an eve service

Once the schema has defined, cloudmesh specifies defaults for managing
a sample data base that is coupled with the REST service. We use
mongodb which could be placed on a sharded mongo service. 

\subsection{Limitations}

The current implementation is a demonstration and showcases that it is
easy to generate a fully functioning REST service based on the
specifications provided in this document. However, it is expected that
scalability, distribution of services, and other advanced options
need to be addrassed based on application requirements.


classes lessons rest.rst
\input{rest}
classes lesson python cmd5.rst
\section{CMD5}\label{cmd5}

CMD is a very useful package in python to create command line shells.
However it does not allow the dynamic integration of newly defined
commands. Furthermore, addition to cmd need to be done within the same
source tree. To simplify developping commands by a number of people and
to have a dynamic plugin mechnism, we developed cmd5. It is a rewrite on
our ealier effords in cloudmesh and cmd3.

\subsection{Resources}\label{resources}

The source code for cmd5 is located in github:

\begin{itemize}
\tightlist
\item
  \url{https://github.com/cloudmesh/cmd5}
\end{itemize}

Installation from source -----------------------

We recommend that you use a virtualenv either with virtualenv or pyenv.
This can be either achieved vor virtualenv with:

\begin{verbatim}
virtualenv ~/ENV2
\end{verbatim}

or for pyenv, with:

\begin{verbatim}
pyenev virtualenv 2.7.13 ENV2
\end{verbatim}

Now you need to get two source directories. We assume yo place them in
\textasciitilde{}/github:

\begin{verbatim}
mkdir ~/github
cd ~/github

git clone https://github.com/cloudmesh/common.git
git clone https://github.com/cloudmesh/cmd5.git
git clone https://github.com/cloudmesh/extbar.git

cd ~/github/common
python setup.py install
pip install .

cd ~/github/cmd5
python setup.py install
pip install .

cd ~/github/extbar
python setup.py install
pip install .
\end{verbatim}

The cmd5 repository contains the shell, while the extbar directory
contains the sample to add the dynamic commands foo and bar.

\subsection{Execution}\label{execution}

To run the shell you can activate it with the cms command. cms stands
for cloudmesh shell:

\begin{verbatim}
(ENV2) $ cms
\end{verbatim}

It will print the banner and enter the shell:

\begin{verbatim}
+-------------------------------------------------------+
|   ____ _                 _                     _      |
|  / ___| | ___  _   _  __| |_ __ ___   ___  ___| |__   |
| | |   | |/ _ \| | | |/ _` | '_ ` _ \ / _ \/ __| '_ \  |
| | |___| | (_) | |_| | (_| | | | | | |  __/\__ \ | | | |
|  \____|_|\___/ \__,_|\__,_|_| |_| |_|\___||___/_| |_| |
+-------------------------------------------------------+
|                  Cloudmesh CMD5 Shell                 |
+-------------------------------------------------------+

cms>
\end{verbatim}

To see the list of commands you can say

\begin{quote}
cms\textgreater{} help
\end{quote}

To see the manula page for a specific command, please use:

\begin{verbatim}
help COMMANDNAME
\end{verbatim}

\subsection{Create your own Extension}\label{create-your-own-extension}

One of the most important features of CMD5 is its ability to extend it
with new commands. This is done via packaged name spaces. This is
defined in the setup.py file of your enhancement. The best way to create
an enhancement is to take a look at the code in

\begin{itemize}
\tightlist
\item
  \url{https://github.com/cloudmesh/extbar.git}
\end{itemize}

Simply copy the code and modify the bar and foo commands to fit yor
needs.

\begin{description}
\item[make sure you are not copying the .git directory. Thus we]
recommend that you copy it explicitly file by file or directory by
directory
\end{description}

It is important that all objects are defined in the command itself and
that no global variables be use in order to allow each shell command to
stand alone. Naturally you should develop API libraries outside of the
cloudmesh shell command and reuse them in order to keep the command code
as small as possible. We place the command in:

\begin{verbatim}
cloudmsesh/ext/command/COMMANDNAME.py
\end{verbatim}

An example for the bar command is presented at:

\begin{itemize}
\tightlist
\item
  \url{https://github.com/cloudmesh/extbar/blob/master/cloudmesh/ext/command/bar.py}
\end{itemize}

It shows how simple the command definition is (bar.py):

\begin{verbatim}
from __future__ import print_function
from cloudmesh.shell.command import command
from cloudmesh.shell.command import PluginCommand

class BarCommand(PluginCommand):

    @command
    def do_bar(self, args, arguments):
        """
        ::
          Usage:
                command -f FILE
                command FILE
                command list
          This command does some useful things.
          Arguments:
              FILE   a file name
          Options:
              -f      specify the file
        """
        print(arguments)
\end{verbatim}

An important difference to other CMD solutions is that our commands can
leverage (besides the standrad definition), docopts as a way to define
the manual page. This allows us to use arguments as dict and use simple
if conditions to interpret the command. Using docopts has the advantage
that contributors are forced to think about the command and its options
and document them from the start. Previously we used not to use docopts
and argparse was used. However we noticed that for some contributions
the lead to commands that were either not properly documented or the
developers delivered ambiguous commands that resulted in confusion and
wrong ussage by the users. Hence, we do recommend that you use docopts.

The transformation is enabled by the @command decorator that takes also
the manual page and creates a proper help message for the shell
automatically. Thus there is no need to introduce a sepaarte help method
as would normally be needed in CMD.

\subsection{Excersise}\label{excersise}

\begin{description}
\item[CMD5.1:]
Install cmd5 on your computer.
\item[CMD5.2:]
Write a new command with your firstname as the command name.
\item[CMD5.3:]
Write a new command and experiment with docopt syntax and argument
interpretation of the dict with if conditions.
\item[CMD5.4:]
If you have useful extensions that you like us to add by default, please
work with us.
\end{description}




\end{document}



