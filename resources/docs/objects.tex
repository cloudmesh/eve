\documentclass[9pt,twocolumn,twoside]{styles/osajnl} 
\usepackage{fancyvrb}
\journal{cm}  

%\usepackage{xcolor}
%usepackage{fullpage}
%\usepackage{fancyvrb}
%\renewcommand{\familydefault}{\sfdefault}
%\usepackage[scaled=0.92]{helvet}
%\usepackage[helvet]{sfmath}
%\everymath={\sf}
%\parindent 0pt

\newcommand{\tightlist}{}

\title{Cloudmesh REST Interface for Virtual Clusters} 

\author[1,*]{Gregor von Laszewski} 
\author[1]{Fugang Wang}
\author[1]{Badi Abdhul-Wahid}

\affil[1]{School of Informatics and Computing, Bloomington, IN 47408, U.S.A.} 
\affil[*]{Corresponding authors: laszewski@gmal.com} 

\dates{Draft v0.0.1, \today} 

\ociscodes{CLoudmesh, REST, NIST} 

\doi{\url{https://github.com/cloudmesh/rest/tree/master/resources/docs}} 


\begin{abstract}

This document summarizes a number of objects that are instrumental for
the interaction with Clouds, Containers, and HPC systems to manage
virtual clusters. 
TBD

\end{abstract}

\begin{document}


\flushbottom % Makes all text pages the same height

\maketitle % Print the title and abstract box

\tableofcontents % Print the contents section
\maketitle



\section{Contribution}

We invite you to contribute to this paper and its discussion to
improve it. Improvements can be done with pull requests. We suggest
you do {\em small} individual changes to a single section and object
rather than large changes as this allows us to integrate the changes
individually and comment on your contribution via github.

Once contributed we will appropriately acknoledge you either as
contributor or author. Please discuss with us how we best acknowledge
you.

\section{Using the Cloudmesh REST Service} 

Components are written as YAML markup in files in the
\verb+resources/samples+ directory.

For example:

\VerbatimInput{../samples/profile.yml}

\subsubsection{Components}

Each resource should have a \verb+description+ entry to act as
documentation. The documentation should be formated as  reStructuredText.

For example:

\begin{Verbatim}
foo:
  description: |
    title description
    =================

    Parameters
    ----------

    - bar: what bar is for
    - baz: what baz is for

  bar: 42
  baz: a string
\end{Verbatim}


\subsubsection{Generating}

Run the \verb+elements+ command:

\begin{verbatim}
cms admin elements <directory> <out.json>
\end{verbatim}

where

\begin{itemize}
\item \verb+<directory>+: directory where the YAML definitions reside
\item \verb+<out.json>+: path to the combined definition
\end{itemize}


For example:

\begin{verbatim}
cms elements resources/samples all.json
\end{verbatim}


\subsubsection{Generating service}

With evegenie installed, the generated JSON file from the above step
is processed to create the stub REST service definitions.




\section{User}

\subsection{Profile}

\VerbatimInput{../samples/profile.yml}
This object defines a user profile.

\subsection{User}

\VerbatimInput{../samples/user.yml}
This defines a user object. Besides the profile information, this may have
other necessary information, e.g., passwrod hash, roles, etc., to facilitate
the Authentication/Authorization.

\section{Cluster}

\subsection{Cluster}

\VerbatimInput{../samples/cluster.yml}
The cluster object has name, label, endpoint and provider. The \textit{endpoint}
defines.... The \textit{provider} defines the nature of the cluster,
e.g., its a virtual cluster on an openstack cloud, or from AWS, or a bare-metal
cluster.


\subsection{Compute Resource}

\VerbatimInput{../samples/compute_resource.yml}
\textbf{compute\_resource} object has attribute \textit{endpoint} which
specifies ... The \textit{kind} could be \textit{baremetal} or \textit{VC}.


\subsection{Compute Resource Kind}

\VerbatimInput{../samples/compute_resource_kind.yml}
This object defines the different compute resource kinds.

\subsection{Computer}

\VerbatimInput{../samples/computer.yml}
This defines a \textbf{computer} object. A computer has name, label,
IP address. It also listed the relevant specs such as memory, disk size, etc.

\subsection{Container}

\VerbatimInput{../samples/container.yml}
This defines \textbf{container} object.


\subsection{Mesos}

\VerbatimInput{../samples/mesos.yml}


\subsection{Kubernetes}

\VerbatimInput{../samples/kubernetes.yml}

\section{Data}

\subsection{File}

\VerbatimInput{../samples/file.yml}
The \textbf{file} object has \textit{name}, \textit{endpoint} (location), \textit{size}
in GB, MB, Byte, \textit{checksum} for integrity check, and last \textit{accessed} timestamp.


\subsection{File Alias}

\VerbatimInput{../samples/file_alias.yml}
A file could have one alias or even multiple ones.

\subsection{Database}

\VerbatimInput{../samples/database.yml}
A \textbf{database} could have a name, an \textit{endpoint} (e.g., host:port),
and protocol used (e.g., SQL, mongo, etc.).


\subsection{Database Protocol}

\VerbatimInput{../samples/database_protocol.yml}
The various database type/protocol supported.


\section{Deployment}

\subsection{Deployment}

\VerbatimInput{../samples/deployment.yml}
A \textbf{deployment} consists of the resource \- \textit{cluster},
the location \- \textit{provider}, e.g., AWS, OpenStack, etc., and
software \textit{stack} to be deployed (e.g., hadoop, spark).


\subsection{Hadoop}

\VerbatimInput{../samples/hadoop.yml}
A \textbf{hadoop} definition defines which \textit{deployer} to be used,
the \textit{parameters} of the deployment, and the system packages as
\textit{requires}. For each requirement, it could have attributes such
as the library origin, version, etc.


\subsection{Mapreduce}

\VerbatimInput{../samples/mapreduce.yml}
This defines a \textbf{mapreduce} deployment with its layered components.

\subsection{Microservice}

\VerbatimInput{../samples/microservice.yml}
A system could be composed of from various microservices, and this defines
each of them.

\subsection{Node}

\VerbatimInput{../samples/node.yml}
A node is composed of multiple components:

\begin{enumerate}
\item Metadata such as the \verb|name| or \verb|owner|.
\item Physical properties such as \verb|cores| or \verb|memory|.
\item Configuration guidance such as \verb|create_external_ip|,
  \verb|security_groups|, or \verb|users|.
\end{enumerate}

The metadata is associated with the node on the provider end (if
supported) as well as in the database. Certain parts of the metadata
(such as \verb|owner|) can be used to implement access
control. Physical properties are relevant for the initial allocation
of the node. Other configuration parameters control and further
provisioning.

In the above, after allocation, the node is configured with a user
called \verb|hello| who is part of the \verb|wheel| group whose
account can be accessed with several SSH identities whose public keys
are provided (in \verb|authorized_keys|).

Additionally, three ssh keys are generated on the node for the
\verb|hello| user. The first uses the \verb|ed25519| cryptographic
method with a password read in from a GPG-encrypted file on the
Command and Control node. The second is a 4098-bit RSA key also
password-protected from the GPG-encrypted file. The third key is
copied to the remote node from an encrypted file on the Command and
Control node.

This definition also provides a security group to control access to
the node from the wide-area-network. In this case all ingress and
egress TCP and UDP traffic is allowed provided they are to ports 22
(SSH), 443 (SSL), and 80 and 8080 (web).


\subsection{Replica}

\VerbatimInput{../samples/replica.yml}
The \textbf{replica} defines a replica for a file.

See also \textbf{file} and \textbf{file\_alias}.


\subsection{Virtual Cluster}

\VerbatimInput{../samples/virtual_cluster.yml}
A \textbf{virtual\_cluster} has \textit{name}, \textit{endpoint}(?), and the \textit{nodes}.

\subsection{virtual compute node}

\VerbatimInput{../samples/virtual_compute_node.yml}
See also \textbf{virtual\_machine}.


\subsection{virtual directory}

\VerbatimInput{../samples/virtual_directory.yml}
The \textbf{virtual\_directory} object defines a virtual directory which has
a name, a \textit{collection} of files, the \textit{endpoint} indicating the
location and where to access, and the \textit{protocol} how to access it.

\subsection{virtual machine}

\VerbatimInput{../samples/virtual_machine.yml}

section{Network}

We are looking for volunteers to contribute here.

\appendix

{|bf README.rst}
\section{Cloudmesh Rest}\label{s:cloudmesh-rest}

Cloudmesh Resst is a refernce implementattion for the NBDRA. It
allows to define automatically a REST service based on the objects
specified by the NBDRA document. In collaboration with other cloudmesh
components it allows easy interaction with hybrid clouds and the
creation of user managed big data services. 

\subsection{Prerequistis}\label{prerequistis}

The preriquisits for Cloudmesh REST are Python 2.7.13 or 3.6.1
it can easily be installed on a variety of systems (at this time we
have only tried ubuntu greater 16.04 and OSX Sierra. However, it would
naturally be possible to also port it to Windows. The instalation
instruction in this document are not complete and we recommend to
refer to the cloudmesh manuals which are under development. The goal
will be to make the instalation (after your system is set up for
developing python) as simple as 

\begin{verbatim}
    pip install cloudmesh.rest
\end{verbatim}


\subsection{REST Service}\label{cm-rest}

With the cloudmesh REST framework it is easy to create REST services
while defining the resources via example json objects. This is achieved
while leveraging the python eve \cite{www-eve} and a modified version of python
evengine \cite{www-cloudmesh-eveengine}. 

A valid json resource specification looks like this:

\begin{verbatim}
{
  "profile": {
    "description": "The Profile of a user",
    "email": "laszewski@gmail.com",
    "firstname": "Gregor",
    "lastname": "von Laszewski",
    "username": "gregor"
  }
}
\end{verbatim}

here we define an object called profile, that contains a number of
attributes and values. The type of the values are automatically
determined. All json specifications are contained in a directory and
can easily be converted into a valid schema for the eve rest service
by executing the commands

\begin{verbatim}
cms schema cat . all.json
cms schema convert all.json
\end{verbatim}

This will create a the configuration \verb|all.settings.py| that can
be used to start an eve service

Once the schema has defined, cloudmesh specifies defaults for managing
a sample data base that is coupled with the REST service. We use
mongodb which could be placed on a sharded mongo service. 

\subsection{Limitations}

The current implementation is a demonstration and showcases that it is
easy to generate a fully functioning REST service based on the
specifications provided in this document. However, it is expected that
scalability, distribution of services, and other advanced options
need to be addrassed based on application requirements.


classes lessons rest.rst
\input{rest}
classes lesson python cmd5.rst
\section{CMD5}\label{cmd5}

CMD is a very useful package in python to create command line shells.
However it does not allow the dynamic integration of newly defined
commands. Furthermore, addition to cmd need to be done within the same
source tree. To simplify developping commands by a number of people and
to have a dynamic plugin mechnism, we developed cmd5. It is a rewrite on
our ealier effords in cloudmesh and cmd3.

\subsection{Resources}\label{resources}

The source code for cmd5 is located in github:

\begin{itemize}
\tightlist
\item
  \url{https://github.com/cloudmesh/cmd5}
\end{itemize}

Installation from source -----------------------

We recommend that you use a virtualenv either with virtualenv or pyenv.
This can be either achieved vor virtualenv with:

\begin{verbatim}
virtualenv ~/ENV2
\end{verbatim}

or for pyenv, with:

\begin{verbatim}
pyenev virtualenv 2.7.13 ENV2
\end{verbatim}

Now you need to get two source directories. We assume yo place them in
\textasciitilde{}/github:

\begin{verbatim}
mkdir ~/github
cd ~/github

git clone https://github.com/cloudmesh/common.git
git clone https://github.com/cloudmesh/cmd5.git
git clone https://github.com/cloudmesh/extbar.git

cd ~/github/common
python setup.py install
pip install .

cd ~/github/cmd5
python setup.py install
pip install .

cd ~/github/extbar
python setup.py install
pip install .
\end{verbatim}

The cmd5 repository contains the shell, while the extbar directory
contains the sample to add the dynamic commands foo and bar.

\subsection{Execution}\label{execution}

To run the shell you can activate it with the cms command. cms stands
for cloudmesh shell:

\begin{verbatim}
(ENV2) $ cms
\end{verbatim}

It will print the banner and enter the shell:

\begin{verbatim}
+-------------------------------------------------------+
|   ____ _                 _                     _      |
|  / ___| | ___  _   _  __| |_ __ ___   ___  ___| |__   |
| | |   | |/ _ \| | | |/ _` | '_ ` _ \ / _ \/ __| '_ \  |
| | |___| | (_) | |_| | (_| | | | | | |  __/\__ \ | | | |
|  \____|_|\___/ \__,_|\__,_|_| |_| |_|\___||___/_| |_| |
+-------------------------------------------------------+
|                  Cloudmesh CMD5 Shell                 |
+-------------------------------------------------------+

cms>
\end{verbatim}

To see the list of commands you can say

\begin{quote}
cms\textgreater{} help
\end{quote}

To see the manula page for a specific command, please use:

\begin{verbatim}
help COMMANDNAME
\end{verbatim}

\subsection{Create your own Extension}\label{create-your-own-extension}

One of the most important features of CMD5 is its ability to extend it
with new commands. This is done via packaged name spaces. This is
defined in the setup.py file of your enhancement. The best way to create
an enhancement is to take a look at the code in

\begin{itemize}
\tightlist
\item
  \url{https://github.com/cloudmesh/extbar.git}
\end{itemize}

Simply copy the code and modify the bar and foo commands to fit yor
needs.

\begin{description}
\item[make sure you are not copying the .git directory. Thus we]
recommend that you copy it explicitly file by file or directory by
directory
\end{description}

It is important that all objects are defined in the command itself and
that no global variables be use in order to allow each shell command to
stand alone. Naturally you should develop API libraries outside of the
cloudmesh shell command and reuse them in order to keep the command code
as small as possible. We place the command in:

\begin{verbatim}
cloudmsesh/ext/command/COMMANDNAME.py
\end{verbatim}

An example for the bar command is presented at:

\begin{itemize}
\tightlist
\item
  \url{https://github.com/cloudmesh/extbar/blob/master/cloudmesh/ext/command/bar.py}
\end{itemize}

It shows how simple the command definition is (bar.py):

\begin{verbatim}
from __future__ import print_function
from cloudmesh.shell.command import command
from cloudmesh.shell.command import PluginCommand

class BarCommand(PluginCommand):

    @command
    def do_bar(self, args, arguments):
        """
        ::
          Usage:
                command -f FILE
                command FILE
                command list
          This command does some useful things.
          Arguments:
              FILE   a file name
          Options:
              -f      specify the file
        """
        print(arguments)
\end{verbatim}

An important difference to other CMD solutions is that our commands can
leverage (besides the standrad definition), docopts as a way to define
the manual page. This allows us to use arguments as dict and use simple
if conditions to interpret the command. Using docopts has the advantage
that contributors are forced to think about the command and its options
and document them from the start. Previously we used not to use docopts
and argparse was used. However we noticed that for some contributions
the lead to commands that were either not properly documented or the
developers delivered ambiguous commands that resulted in confusion and
wrong ussage by the users. Hence, we do recommend that you use docopts.

The transformation is enabled by the @command decorator that takes also
the manual page and creates a proper help message for the shell
automatically. Thus there is no need to introduce a sepaarte help method
as would normally be needed in CMD.

\subsection{Excersise}\label{excersise}

\begin{description}
\item[CMD5.1:]
Install cmd5 on your computer.
\item[CMD5.2:]
Write a new command with your firstname as the command name.
\item[CMD5.3:]
Write a new command and experiment with docopt syntax and argument
interpretation of the dict with if conditions.
\item[CMD5.4:]
If you have useful extensions that you like us to add by default, please
work with us.
\end{description}


\end{document}



