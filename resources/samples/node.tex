A node is composed of multiple components:

\begin{enumerate}
\item Metadata such as the \verb|name| or \verb|owner|.
\item Physical properties such as \verb|cores| or \verb|memory|.
\item Configuration guidance such as \verb|create_external_ip|,
  \verb|security_groups|, or \verb|users|.
\end{enumerate}

The metadata is associated with the node on the provider end (if
supported) as well as in the database. Certain parts of the metadata
(such as \verb|owner|) can be used to implement access
control. Physical properties are relevant for the initial allocation
of the node. Other configuration parameters control and further
provisioning.

In the above, after allocation, the node is configured with a user
called \verb|hello| who is part of the \verb|wheel| group whose
account can be accessed with several SSH identities whose public keys
are provided (in \verb|authorized_keys|).

Additionally, three ssh keys are generated on the node for the
\verb|hello| user. The first uses the \verb|ed25519| cryptographic
method with a password read in from a GPG-encrypted file on the
Command and Control node. The second is a 4098-bit RSA key also
password-protected from the GPG-encrypted file. The third key is
copied to the remote node from an encrypted file on the Command and
Control node.

This definition also provides a security group to control access to
the node from the wide-area-network. In this case all ingress and
egress TCP and UDP traffic is allowed provided they are to ports 22
(SSH), 443 (SSL), and 80 and 8080 (web).
